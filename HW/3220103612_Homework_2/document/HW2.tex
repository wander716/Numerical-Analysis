\documentclass{article}
% 设置页面的环境,a4纸张大小,左右上下边距信息
\usepackage[a4paper,left=10mm,right=10mm,top=15mm,bottom=15mm]{geometry}
\usepackage[UTF8]{ctex}
\usepackage{amsmath}
\usepackage{graphicx} % Required for inserting images
\usepackage{subfigure}

\title{Numerical Analysis HW2}
\author{3220103612 章杨}
\date{October 2023}

\begin{document}
\maketitle
\large
\section{Problem1}
选取精度为0.00001,得到$p_2=-0.8654740331$。\\

若采取$p_0=0$,根据$p_n=p_{n-1}-\frac{f(x)}{f^{'}(x)}$,$f^{'}(p_0)=f^{'}(0)=0$,故不能作为分母,因此不能使用牛顿方法

\section{Problem2}
\subsection{(i)}
$f(x)=b-\frac{1}{x},f^{'}(x)=\frac{1}{x^2}$,可得

$$    x_{n+1}=2x_n-bx^{2}_n $$

$$\epsilon_k=\frac{\frac{1}{b}-x_k}{\frac{1}{b}}=1-bx_k$$
$$\epsilon_{k+1}=\frac{\frac{1}{b}-x_{k+1}}{\frac{1}{b}}=\frac{\frac{1}{b}-2x_k+bx^2_k}{\frac{1}{b}}=(1-bx_k)^2$$
$$ \epsilon_{k+1}=\epsilon_k^2$$
\subsection{(ii)}
对任意$x_n\in (0,\frac{2}{b})$,$x_n+1 =x_n(2-bx_n)>0$(数学归纳法),所以存在$x_1\in(0,\frac{1}{b})$与k$\in(0,1)$,使得
$$g(x)=2x-bx^2$$
$$g^{'}(x)=2-2bx_1=2(1-bx_1)<k$$
根据收敛性定理,x收敛于$\frac{1}{b}$

\section{Problem 3}
代码附在code文件夹中
\subsection{a.}
$x_2$ =  [ 0.50016669  0.25080364 -0.51738743]
\subsection{b.}
$x_2$ =  [  4.35087719  18.49122807 -19.84210526]
\section{Problem 4}
代码附在code文件夹中
\subsection{a.}
[1.04122820973906, 1.06586158726298, 0.928075247526470]
此时,g(x)=0.0361977296438597
\subsection{b.}
[0.510630019323346, 1.01246048957044, -0.468346515234894],
此时,g(x)=0.0487719856095837\\

过程中发现精确解为[0.5,1,-0.5]
\section{Problem 5}
\subsection{a.}
关注四个偏导函数,
$$\frac{ \partial g_1 }{ \partial x_1 }=\frac{x_1}{5}\leq0.3$$
$$\frac{ \partial g_1 }{ \partial x_2 }=\frac{x_2}{5}\leq0.3$$
$$\frac{ \partial g_2 }{ \partial x_1 }=\frac{x_2^2 +1}{10}\leq0.325$$
$$\frac{ \partial g_2 }{ \partial x_2 }=\frac{x_1 x_2}{5}\leq0.45$$
取$K=0.91$,则$$|\frac{ \partial g_i }{ \partial x_i }|\leq \frac{K}{n}$$

\subsection{b.}
$$x^{(0)}=(0,1)^t$$
$$x^{(1)}=(0.9,0.8)^t$$
$$x^{(2)}=(1.045,0.9046)^t$$
\end{document}

